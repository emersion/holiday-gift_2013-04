\documentclass[12pt]{article}

\usepackage[utf8]{inputenc}
\usepackage[T1]{fontenc}
\usepackage[francais]{babel}

\usepackage{amsmath}
\usepackage{amssymb}
\usepackage{mathrsfs}

\title{\textbf{Activités}}
\date{}
\begin{document}

\maketitle

\section{}
\section{Activité 2}

Démontrer que $\overrightarrow{u} . \overrightarrow{v} = ||\overrightarrow{u}|| \times ||\overrightarrow{v}|| \times \cos(\overrightarrow{u}, \overrightarrow{v})$.

Cf. fig. 2.

On pose :
\[
\overrightarrow{AB} = \overrightarrow{u}\\
\overrightarrow{AC} = \overrightarrow{v}\\
(\overrightarrow{u};\overrightarrow{v}) = (\overrightarrow{AB};\overrightarrow{AC})
\]

Soit $\overrightarrow{i}$ tel que $\overrightarrow{AB} = AB \times \overrightarrow{i}$ et $\overrightarrow{j}$ tel que $(A;\overrightarrow{i};\overrightarrow{j})$soit orthonormé. Le cercle orienté de centre $A$ et de rayon 1 est donc le cercle trigonométrique.

\[
AK = 1 = ||\overrightarrow{i}|| = ||\overrightarrow{j}||\\
\overrightarrow{AC} = AC \times \overrightarrow{AK}
\]

On a donc :

\[
K(\cos \theta;\sin \theta)\\
\overrightarrow{AC}(AC \cos \theta;AC \sin \theta)\\
\overrightarrow{AB}(AB;0)
\]

\[
\overrightarrow{AB}.\overrightarrow{AC} = AB \times AC \times \cos(\overrightarrow{u};\overrightarrow{v})
\]

Or $\cos(\overrightarrow{u};\overrightarrow{v})=\cos(\overrightarrow{v};\overrightarrow{u})$.

\[
\cos(\overrightarrow{AB};\overrightarrow{AC})=\cos(\overrightarrow{AC};\overrightarrow{AB})=\cos \widehat{BAC}
\]

\textbf{Conséquence} : exprimer $\overrightarrow{u}.\overrightarrow{v}$ dans le cas $\overrightarrow{u}$ et $\overrightarrow{v}$ colinéaires.

$\overrightarrow{u}$ et $\overrightarrow{v}$ colinéaires $\Leftrightarrow$ $(u;v)=0 [2\pi]$\footnote{Même sens} ou $(u;v)=\pi [2\pi]$\footnote{Sens contraire}.

On en déduit :
\[
\overrightarrow{u}.\overrightarrow{v} = ||\overrightarrow{u}|| \times ||\overrightarrow{v}|| \times \cos 0 = ||\overrightarrow{u}|| \times ||\overrightarrow{v}||
\]
ou
\[
\overrightarrow{u}.\overrightarrow{v} = ||\overrightarrow{u}|| \times ||\overrightarrow{v}|| \times \cos \pi = - ||\overrightarrow{u}|| \times ||\overrightarrow{v}||
\]

\section{Activité 3}

Soient $A$, $B$, $C$ non alignés, $H$ le projeté orthogonal de $C$ sur $(AB)$.

Démontrer que $\overrightarrow{AB}.\overrightarrow{AC} = \overrightarrow{AB}.\overrightarrow{AH}$.

\end{document}