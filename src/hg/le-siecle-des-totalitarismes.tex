\documentclass[12pt]{article}

\usepackage[utf8]{inputenc}
\usepackage[T1]{fontenc}
\usepackage[francais]{babel}

\renewcommand{\emph}{\textbf}

\title{\textbf{Le siècle des totalitarismes}}
\date{}
\begin{document}

\maketitle

\section{}
\section{}
\section{L'idéologie dans les trois totalitarismes européens}

\subsection{L'ère des masses}

\textit{Voir votre cours...}

\subsection{Les idéologies fascistes}

\paragraph{}
Le fascisme et le nazisme ont comme base commune le \emph{nationalisme}. Mussolini souhaite reconstruire autour de la Méditerranée un \emph{empire} équivalent à celui de Rome dans l'antiquité.

\paragraph{}
Hitler pensait qu'il fallait aux allemands un espace vital, le \emph{Lebensraum}. Dans cette optique, les nazis imaginaient la conquête de l'Europe orientale (Pologne, URSS) et la déportation des peuples qui y vivaient pour installer des \emph{colons allemands}.

\paragraph{}
Contrairement au fascisme italien, le nazisme ajoute au nationalisme \emph{le racisme et l'antisémitisme}. Les nazis considéraient que le peuple allemand était de \emph{race aryenne}, c'est-à-dire supérieure à toute autre. L'archétype de l'aryen est un européen blond aux yeux bleus. Les nazis souhaitèrent que cette \textit{race} soit parfaite. Une \emph{politique nataliste} a été conçue pour obtenir des enfants aryens. Ils ont aussi pratiqué l'\emph{eugénisme}, c'est-à-dire l'élimination des individus considérés comme non représentants de la race (handicapés, homosexuels avec élimination ou stérilisation). Dans \textit{Mein Kampf}, Hitler définit une hiérarchie dans les \textit{groupes} humains. Si les allemands sont dans la race des seigneurs, les slaves sont considérés comme des races \textit{inférieures}. Les Juifs sont à part. En 1935, les \emph{lois de Nuremberg} excluent les Juifs de la société allemande\footnote{Doc. 4 page 205}.

\subsection{L'idéologie dans l'URSS de Staline}

L'URSS se définit lui-même comme \emph{communiste}, c'est-à-dire d'inspiration marxiste. Karl Marx était un théoricien. C'est \emph{Lénine} qui a adapté aux événements la théorie marxiste. On parle donc de marxisme léninisme.

\paragraph{}
Le marxisme repose sur l'idée que la société ne doit plus se composer de classes antagonistes. Le bonheur doit être obtenu en donnant à chacun les mêmes moyens pour vivre. Cette \emph{doctrine égalitaire} revendique aussi une égalité entre les sexes.\\
Le léninisme arguait\footnote{Revendiquer, donner comme argument majeur}  que le prolétariat était représenté par son élite : son parti. De ce fait, la dictature communiste se veut \emph{démocratique} (au nom du peuple). Staline n'a pas changé sur le plan théorique ce qu'avait laissé en place Lénine.

\paragraph{}
En 1921, pour faire face à la crise économique qui frappait la Russie, Lénine avait conçu la NEP (nouvelle politique économique). Les paysans gardaient la propriété de leurs terres et de leur production. En 1929, au nom du communisme, Staline décide de \emph{collectiviser l'agriculture}. Les terres sont nationalisées, les outils et les productions sont confisqués. Les paysans ont l'obligation d'intégrer les \textit{Kolkhozes}.

\section{Le fonctionnement des totalitarismes}

\subsection{Des dictatures}

\paragraph{}
Dans les trois États totalitaires, le pouvoir est détenu par un \emph{chef charismatique} à qui un culte de la personnalité est rendu. Il s'agit de Staline qui n'a pas eu d'autre titre que secrétaire général du Parti Communiste (1929-1953), de Mussolini qui a pris le nom de Duce et d'Hitler nommé Führer.

\paragraph{}
Ces chefs charismatiques utilisent la force d'un \emph{parti unique} dans chacun des pays. Il s'agit du Parti Communiste (PC), le PNF et le NSDAP.\\
En URSS, le PC domine l'État. En Italie, le PNF est à la tête de l'État, mais il reste dans l'administration des fonctionnaires qui n'en font pas partie : le Duce gouverne mais le chef de l'État reste le roi et beaucoup de fonctionnaires y restent fidèles. Dans la mesure où il existe encore un contre-pouvoir incarné par l'Eglise, le fascisme italien est donc un totalitarisme imparfait\footnote{Quelqu'un peut encore s'opposer au régime}.\\
En Allemagne, Hitler sépare le NSDAP et l'État tout en contrôlant l'un et l'autre. Il en découle une série de rivalités et de concurrence entre les deux organismes.

\paragraph{}
En URSS, le pouvoir étant détenu par le parti, le fonctionnement du régime repose sur la bureaucratie\footnote{\textit{Tout le monde fait des petits papiers (rapports) contre tout le monde}}. Quand les décisions de Staline aboutissent à des résultats négatifs et une la critique peut déséquilibrer son pouvoir, Staline provoque des \emph{purges}, c'est-à-dire l'élimination des cadres du parti qui peuvent menacer le dictateur. Exemple : les grands procès de Moscou (1938-1940).\\
En Allemagne, Hitler ne réunit pas régulièrement son gouvernement. Il donne des ordres et chaque ministre cherche à les appliquer en fonction de ses intérêts. Ce fonctionnement aboutit à une \emph{très grande incohérence} (avec des ordres, contre-ordres, ordres secrets...).

\subsection{Des régimes marqués par la terreur}

\paragraph{}
Chaque régime totalitaire a sa \emph{police politique} et ses \emph{camps de concentration}. En URSS, il y a le GPU et le NKVD. En Italie, c'est l'OVRA. En Allemagne, la GESTAPO doublée de la SS.\\
Les moyens déployés contre la population sont considérables : surveillance, appels à la délation, poursuites, arrestations, torture, sanctions pouvant aller de l'exécution à la déportation.

\paragraph{}
Chaque régime définit ses ennemis :
\begin{itemize}
\item En URSS, il s'agit des \emph{ennemis de classe} : les bourgeois, les propriétaires de terres, les \textit{pope}\footnote{Prêtres, moines... (\textit{"La religion est l'opium du peuple"})}. A ces ennemis il faut ajouter tous les étrangers qui ont montré leur hostilité au régime. Ce dernier aspect aboutit à une méfiance à l'égard des communistes.
\item En Italie et en Allemagne, l'ennemi est avant tout \emph{politique}. En premier lieu, les communistes, socialistes, démocrates, syndicalistes, tous ceux qui critiquent le régime.
\item L'Allemagne a une particularité : un \emph{antisémitisme} virulent. Cet antisémitisme s'exprime pleinement en 1938 lors de la \emph{nuit de cristal} : les Nazis attaquent les magasins juifs et les synagogues, les incendient. Il y eut les blessés et des Juifs assassinés. Le lendemain, l'État allemand condamne la communauté juive pour trouble à l'ordre public. Les Juifs doivent payent une amende colossale et beaucoup d'entre eux sont enfermés à Dachau. Dans le discours nazi, l'ennemi racial et l'ennemi politique ne forment qu'un : ils parlent des \textit{judéobolchéviques}.
\end{itemize}

\paragraph{}
Il existe des différences de degrés entre les trois régimes en terme de \emph{répression}. En Italie, les prisonniers politiques représentaient relativement peu de monde (à peu près 6 000 hommes) et il n'y a pas eu de volonté d'extermination systématique.\\
En URSS, la répression a porté sur des \emph{masses importantes} de citoyens qui se comptent en millions. Lors de la collectivisation des terres, une partie importante des paysans a résisté. Le régime a alors activé la détresse politique, en premier lieu la propagande, en qualifiant de Koulak\footnote{Paysan riche} ceux qui résistaient. La \emph{dékoulakisation} a pris la forme d'exécutions sommaires, de déportations au Goulag et même de famines organisées par l'État, par exemple en Ukraine. Dans les années 1936-1938, le régime qui connaît des difficultés intérieures dues à la modernisation de l'économie rejette sa responsabilité sur des comportements de traîtres. La répression s'est abattue sur le population et a envoyé au Goulag des centaines de milliers de citoyens. Cette répression s'est aussi abattue sur les membres du parti communiste dont les critiques auraient pu être une menace pour le pouvoir personnel de Staline. Des grands procès sont organisés à Moscou, qui condamnent à mort tous les anciens dirigeants du parti bolchevique, mais aussi l'essentiel des chefs de l'armée.\\
En Allemagne, la répression s'est abattue sur les opposants et sur les Juifs. Durant la guerre, elle a dépassé tous les degrés de l'horreur.

\subsection{L'encadrement des sociétés}

\paragraph{}
La population est soumise à une \emph{propagande} constante. Chaque État totalitaire se dote d'un ministère de la propagande. Tous les moyens sont bons : l'école mais aussi les médias, la presse, les affiches, les livres, l'art\footnote{Cf. doc. 3 p. 201}, la radio et le cinéma. Cette propagande utilise aussi la \emph{censure} : en 1933, d'immenses \textit{autodafés} détruisent d'immenses quantités de livres considérés comme hostiles à la race allemande\footnote{Cf. doc. 2 p. 215} (ouvrages socialistes, syndicalistes, juifs).

\paragraph{}
La population est embrigadée dans des organisations contrôlées par l'État. Il s'agit par exemple des \emph{organisations de jeunesse} : l'\textit{œuvre nationale balilla} en Italie, les \textit{jeunesses hitlériennes} en Allemagne et les \textit{Komsomols} en URSS. L'État s'occupe aussi des \emph{loisirs} : le \textit{Dopolavoro} (Italie) et \textit{La Force par la Joie} (Allemagne) s'occupent des vacances des citoyens. Seule une infime minorité d'italiens ou d'allemands partaient en vacances. En URSS fut instauré le \textit{Dimanche communiste}, soit l'obligation de travailler gratuitement le dimanche pour le bien-être communiste.

\paragraph{}
L'\emph{économie} n'échappe pas à l'emprise du totalitarisme. En URSS, l'intégralité des moyens de production appartient à l'État. Le \textit{Gosplan} (ministère de l'économie) établit des \emph{plans quinquennaux}. À partir de 1929, la priorité est donnée à l'industrie lourde (industrie, énergie, etc...). La propagande mobilise les énergies pour accomplir les plans : des affiches\footnote{Cf. doc. 2 p. 202}, le \emph{stakhanovisme}\footnote{Politique visant à augmenter le rendement du travail industriel. Cf. définition p. 207.}.\\
En Italie et en Allemagne, le régime intervient dans l'économie mais la plupart des entreprises restent des propriétés privées. Le premier objectif dans ces deux régimes est d'aboutir à l'\emph{autarcie}, c'est-à-dire au fait de ne pas être dépendant de l'étranger. En Italie par exemple, la \textit{bataille du blé}\footnote{Cf. doc. 1 p. 215} fut lancée en 1925. En Allemagne, la priorité est la \emph{production d'armes}. Dans les deux régimes fascistes, l'économie doit aussi être aussi service de la \emph{gloire du régime}. Par exemple, l'État construit un réseau moderne d'autoroutes. En Allemagne, le régime a voulu une voiture du peuple. C'est l'ingénieur Ferdinand Porche qui a conçu la \textit{Volkswagen}.

\end{document}