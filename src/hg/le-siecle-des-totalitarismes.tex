\documentclass[12pt]{article}

\usepackage[utf8]{inputenc}
\usepackage[T1]{fontenc}
\usepackage[francais]{babel}

\title{\textbf{Le siècle des totalitarismes}}
\author{Simon Ser}
\date{2013-04-22}
\begin{document}

\section{}
\section{}
\section{L'idéologie dans les trois totalitarismes européens}

\subsection{L'ère des masses}

\emph{Voir votre cours...}

\subsection{Les idéologies fascistes}

Le fascisme et le nazisme ont comme base commune le \textbf{nationalisme}. Mussolini souhaite reconstruire autour de la Méditerrannée un \textbf{empire} équivalent à celui de Rome dans l'antiquité.

Hitler pensait qu'il fallait aux allemands un espace vital, le \textbf{Lebensraum}. Dans cette optique, les nazis imaginaient la conquête de l'Europe orientale (Pologne, URSS) et la déportation des peuples qui y vivaient pour installer des \textbf{colons allemands}.

Contrairement au fascisme italien, le nazisme ajoute au nationalisme \textbf{le racisme et l'antisémitisme}. Les nazis considéraient que le peuple allemand était de \textbf{race aryenne}, c'est-à-dire supérieure à toute autre. L'archétype de l'aryen est un européen blond aux yeux bleus. Les nazis souhaitèrent que cette \emph{race} soit parfaite. Une \textbf{politique nataliste} a été conçue pour obtenir des enfants aryens. Ils ont aussi pratiqué l'\textbf{eugénisme}, c'est-à-dire l'élimination des individus considérés comme non représentants de la race (handicapés, homosexuels avec élimination ou stérilisation). Dans \emph{Mein Kampf}, Hitler définit une hiérarchie dans les \emph{groupes} humains. Si les allemands sont dans la race des seigneurs, les slaves sont considérés comme des races \emph{inférieures}. Les Juifs sont à part. En 1935, les \textbf{lois de Nuremberg} excluent les Juifs de la société allemande\footnote{Doc. 4 page 205}.

\subsection{L'idéologie dans l'URSS de Staline}

L'URSS se définit lui-même comme \textbf{communiste}, c'est-à-dire d'inspiration marxiste. Karl Marx était un théoricien. C'est \textbf{Lénine} qui a adapté aux événements la théorie marxiste. On parle donc de marxisme léninisme.

Le marxisme repose sur l'idée que la société ne doit plus se composer de classes antogonistes. Le bonheur doit être obtenu en donnant à chacun les mêmes moyens pour vivre. Cette \textbf{doctrine égalitaire} revendique aussi une égalité entre les sexes.\\
Le léninisme arguait\footnote{Revendiquer, donner comme argument majeur}  que le prolétariat était représenté par son élite : son parti. De ce fait, la dictature communiste se veut \textbf{démocratique} (au nom du peuple). Staline n'a pas changé sur le plan théorique ce qu'avait laissé en place Lénine.

En 1921, pour faire face à la crise économique qui frappait la Russie, Lénine avait conçu la NEP (nouvelle politique économique). Les paysans gardaient la propriété de leurs terres et de leur production. En 1929, au nom du communisme, Staline décide de \textbf{collectiviser l'agriculture}. Les terres sont nationalisées, les outils et les productions sont confisqués. Les paysans ont l'obligation d'intégrer les \emph{Kolkhozes}.

\section{Le fonctionnement des totalitarismes}

\subsection{Des dictatures}

Dans les trois Etats totalitaires, le pouvoir est détenu par un \textbf{chef charismatique} à qui un culte de la personnalité est rendu.

\end{document}