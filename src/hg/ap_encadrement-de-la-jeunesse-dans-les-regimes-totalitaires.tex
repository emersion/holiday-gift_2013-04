\documentclass[12pt]{article}

\usepackage[utf8]{inputenc}
\usepackage[T1]{fontenc}
\usepackage[francais]{babel}

\renewcommand{\emph}{\textbf}

\title{\textbf{AP - Encadrement de la jeunesse dans les régimes totalitaires}}
\date{}
\begin{document}

\maketitle

\section{Les organisations de jeunesse}

\textit{Voir cours précédant...}

\section{Les objectifs des organisations de jeunesse totalitaires}

\textit{Voir cours précédant...}

\paragraph{3}
Les jeunes deviennent un \emph{outil de propagande} auprès de la société. Ils représentent un exemple de réussite du régime. Les enfants doivent porter l'idéologie au reste de la société.

\section{Les activités dans les organisations de jeunesse}

\paragraph{}
Les activités se font en commun. En URSS, garçons et filles sont mélangés. Les jeunesses communistes suivent une \emph{formation scolaire approfondie} et doivent se soumettre à des activités en faveur du prolétariat, par exemple la lutte contre l'analphabétisme. Les jeunesses communistes font aussi des activités en plein air et ils participent à toutes sortes de \emph{défilés} en faveur du régime.

\paragraph{}
Dans l'Italie fasciste, garçons et filles sont séparés. Les garçons reçoivent une formation essentiellement \emph{militaire}. Certains doivent monter la garde devant les bâtiments officiels. Les filles reçoivent aussi une formation idéologique mais les activités étaient particulières dans la mesure où le régime pensait que les femmes devaient rester à la maison pour s'occuper des enfants et tenir le foyer. Pour séduire les jeunes, des \emph{camps de vacances} étaient organisés.

\paragraph{}
En Allemagne comme en Italie, garçons et filles sont séparés. Le régime pousse beaucoup plus loin la \emph{formation militaire} des jeunesses hitlériennes. Les activités sportives sont plus intenses en Allemagne qu'en Italie. Conformément à l'idéologie nazie, la \emph{virilité} est exaltée.

\end{document}